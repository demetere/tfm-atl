\begin{abstract}
    Traffic congestion in urban areas is a pervasive problem caused by the ever-increasing population and transportation demands. Traditional fixed-time traffic signal systems often result in inefficiencies and unnecessary delays, especially in low traffic density scenarios. This research proposes a novel Traffic Control System based on reinforcement learning (RL) to address these challenges. By incorporating RL, we aim to create an intelligent traffic control system that adapts to real-time traffic conditions, reduces delays, minimizes environmental impact, and enhances road safety. Our study focuses on a 4-way intersection, analyzing incoming traffic density to optimize traffic signal control and improve overall transportation efficiency over time.
    
    The motivation for this research stems from the need to alleviate traffic problems in various regions, including the author's home country, Georgia, and the global demand for intelligent traffic control systems. Rapid urbanization and population growth have placed a substantial burden on urban transportation infrastructure, necessitating innovative and adaptive solutions. By harnessing RL and artificial intelligence, this research aims to contribute to the development of intelligent traffic control systems that can serve as models for cities worldwide.
    
    The primary objectives of this research include the creation of a realistic urban traffic simulation environment, the generation of authentic traffic patterns, the utilization of state-of-the-art RL algorithms, and a comparative analysis of the performance of RL-based controllers against baseline controllers. The research concludes with insights into the potential for intelligent traffic management systems to alleviate congestion, improve efficiency, and reduce environmental impacts.
    
    In summary, this thesis explores the complexities of urban traffic control and the potential of RL-based solutions. It emphasizes the importance of adaptive and context-aware traffic light control systems, offering valuable insights for the optimization of urban traffic flow in real-world settings.
\end{abstract}
    