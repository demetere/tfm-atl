\begin{abstract}
    The ever-increasing population as well as the ever-increasing demands placed on transportation are the root causes of the persistent problem of traffic congestion in metropolitan places. In situations with low traffic density, traditional traffic signal systems typically result in inefficiencies and unnecessary delays. These problems are compounded when the signal times are fixed. In order to overcome these obstacles, the study presented here suggests a unique Traffic Control System that is founded on reinforcement learning (RL). We hope that by adding RL, we will be able to construct an intelligent traffic control system that can adapt to the circumstances of the traffic in real time, lessen the impact on the environment, shorten the length of delays, and improve road safety. Our research focuses on a four-way junction, where we examine the incoming traffic density to determine how best to operate the traffic signals and how we may increase the overall efficiency of transportation over time.
    
    The motivation for this research stems from the need to alleviate traffic problems in various regions, including the author's home country, Georgia, and the global demand for intelligent traffic control systems. Rapid urbanization and population growth have placed a substantial burden on urban transportation infrastructure, necessitating innovative and adaptive solutions. By harnessing RL and artificial intelligence, this research aims to contribute to the development of intelligent traffic control systems that can serve as models for cities worldwide.
    
    The primary objectives of this research include the creation of a realistic urban traffic simulation environment, the generation of authentic traffic patterns, the utilization of state-of-the-art RL algorithms, and a comparative analysis of the performance of RL-based controllers against baseline controllers. The research concludes with insights into the potential for intelligent traffic management systems to alleviate congestion, improve efficiency, and reduce environmental impacts.
    
    In summary, this thesis explores the complexities of urban traffic control and the potential of RL-based solutions. It emphasizes the importance of adaptive and context-aware traffic light control systems, offering valuable insights for the optimization of urban traffic flow in real-world settings.
\end{abstract}
    