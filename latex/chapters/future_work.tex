\chapter{Future Work}
In this chapter, we delve into the exciting realm of potential future endeavors that could further enhance the effectiveness of the research presented in this thesis. Numerous possibilities lie ahead, each holding the promise of enriching our understanding and application of traffic flow optimization.

\section{Enhancing the Simulation Environment}
One avenue for future work involves expanding the scope of our traffic simulation to create a more comprehensive and realistic environment. To achieve this, several key enhancements can be considered.

Firstly, the incorporation of bus simulations into the SUMO (Simulation of Urban MObility) environment stands as a pivotal step. The inclusion of buses introduces an additional layer of complexity to the traffic dynamics. Consequently, a natural progression involves adapting the reward functions to prioritize public transport vehicles, ensuring minimal delays for buses. This adjustment acknowledges the importance of efficient public transportation systems in urban areas and aligns with the broader goal of sustainable urban mobility.

\section{Pedestrian Simulation Integration}
Taking realism to a higher level, the integration of pedestrian simulations into our existing framework represents another compelling avenue. SUMO offers the capability to simulate pedestrian transportation, thereby introducing new variables into the traffic ecosystem. The introduction of pedestrian-specific traffic lights and pathways adds a layer of complexity that challenges traffic light control strategies.

This expansion necessitates the refinement and sophistication of reward functions and state representations. Balancing the needs of vehicles, public transport, and pedestrians becomes a multifaceted optimization problem. The ability of agents to process and respond to this wealth of information effectively becomes a fascinating research endeavor. One potential approach is the creation of distinct agents responsible for specific aspects of traffic control, with an overarching manager making decisions based on their outputs. This hierarchical structure offers a promising direction for handling intricate traffic scenarios.

\section{Exploring Reward System Variations}
Diving deeper into the realm of reinforcement learning, future work could entail the exploration of various reward systems and state representations. The flexibility of reward design allows for the experimentation with alternative combinations of state attributes. However, it is crucial to strike a balance, as overly complex reward structures may challenge the ability of agents to discern the underlying optimization objectives.

Additionally, the examination of pre-existing reward functions designed for similar agents merits consideration. The evaluation of these established reward systems within our framework can provide insights into their adaptability and effectiveness in optimizing traffic light control.

\section{Contributions to the RESCO Repository}
All of these future endeavors should ideally be conducted within the framework of the RESCO repository. By contributing to this public open-source repository, we can not only advance our own research but also benefit the broader scientific community. Improving the RESCO repository with enhanced simulation capabilities, additional traffic elements, and innovative control strategies ensures that the research conducted here serves as a valuable resource for other researchers, fostering collaboration and knowledge exchange.

In conclusion, the future of this research presents an array of exciting opportunities to push the boundaries of traffic flow optimization. These potential avenues offer the promise of more realistic simulations, advanced control strategies, and valuable contributions to the research community.
