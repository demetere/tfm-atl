\chapter{Introduction}
% Overview Section
\section{Overview}
Traffic congestion is a persistent global issue, impeding daily commutes as a result of the ever-increasing urban population and transportation demands in cities worldwide \cite{leveinson1998speed}\cite{tirachini2013estimation}. One major contributor to this problem is the delay caused by red lights at intersections, where traffic signals typically operate on fixed-time schedules regardless of actual traffic conditions \cite{mousavi2017traffic}. While such systems are effective in heavily congested areas, they often prove inefficient for low traffic density scenarios, resulting in unnecessary delays and fuel wastage \cite{mousavi2017traffic}.

Recent technological advancements have introduced the Adaptive Traffic Signal Control System, which utilizes sensors embedded in roads to synchronize traffic signals, thus responding to real-time traffic conditions \cite{leveinson1998speed}. However, this system's feasibility and cost-effectiveness have been questioned due to the need for embedded road infrastructure and power sources \cite{leveinson1998speed}. Additionally, optimizing traffic signal control to minimize delays while ensuring system stability remains a challenge \cite{mousavi2017traffic}.

This thesis aims to address these challenges by proposing a Traffic Control System based on reinforcement learning (RL), an artificial intelligence framework that learns optimal decision policies through continuous adaptation to real-time traffic scenarios. By moving away from fixed-time schedules and incorporating RL, we seek to develop an intelligent traffic control system that efficiently manages traffic flow, reduces environmental impact, such as air pollution and fuel wastage, and enhances road safety \cite{mousavi2017traffic}. The research focuses on a 4-way intersection, analyzing incoming traffic density to optimize traffic signal control and improve overall transportation efficiency over time.

\section{Motivation}
My personal motivation for undertaking this thesis is deeply rooted in the traffic problems I have witnessed in my home country, Georgia. The congestion and inefficiency of traffic lights on some of the busiest streets in Georgia have long been a source of frustration for me and my fellow citizens. The resulting traffic jams not only waste valuable time but also contribute to environmental issues such as increased air pollution and fuel wastage. Additionally, the risk of accidents rises as traffic congestion worsens.

Driven by these challenges, I have chosen to focus on optimizing traffic light control, specifically targeting one of the busiest streets in Georgia, as the subject of my thesis. By harnessing the power of reinforcement learning, I aim to develop an intelligent traffic control system capable of dynamically managing traffic flows, reducing congestion, and addressing environmental concerns. Through this research, I aspire to contribute to more efficient and sustainable urban transportation systems, not only in Georgia but also as a model for cities worldwide.