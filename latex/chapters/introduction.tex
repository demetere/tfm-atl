\chapter{Introduction}
% Overview Section
\section{Overview}
As a result of an ever-increasing urban population and the rising demand for transportation in cities all over the world, traffic congestion is a chronic global issue that impedes everyday commutes. \cite{leveinson1998speed, tirachini2013estimation} This is a problem that affects cities all over the world. The delay that is generated by red lights at junctions, which often run on fixed-time schedules regardless of real traffic circumstances \cite{mousavi2017traffic}, is one of the primary factors that contributes to this problem. Although such systems are useful in locations with a high volume of traffic, experience shows that they are frequently ineffective in settings with a low volume of traffic. As a consequence, they cause needless delays and wasteful use of fuel \cite{mousavi2017traffic}.

In light of recent developments in technology, the Adaptive Traffic Signal Control System has been developed. This system makes use of sensors that are installed in roadways in order to synchronize traffic lights, and it does so in order to adapt to real-time traffic circumstances \cite{leveinson1998speed}. Despite this, the practicability and cost-effectiveness of this system have been called into question because it requires embedded road infrastructure as well as power sources \cite{leveinson1998speed}. In addition, improving traffic signal regulation in order to reduce delays as much as possible while maintaining system reliability continues to be a difficult task \cite{mousavi2017traffic}.

Reinforcement learning (RL) is a framework for artificial intelligence that learns optimum decision policies via continuous adaptation to real-time traffic scenarios. The purpose of this thesis is to solve these issues by developing a Traffic Control System that is based on RL. This system would be used to manage traffic. We want to establish an intelligent traffic control system by moving away from fixed-time schedules and adding RL. This will allow us to efficiently manage traffic flow, decrease environmental impacts like as air pollution and fuel wastage, and improve road safety \cite{mousavi2017traffic}. This study will concentrate on a four-way junction, examining the density of incoming traffic to determine the best strategy to operate the traffic signals and increase the overall efficiency of transportation over time.

% Motivation Section
\section{Motivation}
My personal motivation for embarking on this thesis is deeply rooted in the persistent traffic problems that afflict my home country, Georgia. The congestion and inefficiency of traffic lights on some of the busiest streets in Georgia have long been a source of frustration for me and my fellow citizens. The resulting traffic jams not only waste valuable time but also contribute to environmental issues such as increased air pollution and fuel wastage. Moreover, the heightened risk of accidents in congested traffic conditions underscores the urgency of finding effective solutions.

Beyond my personal experiences, the global need for intelligent traffic control systems has never been more evident. Rapid urbanization and population growth have placed an ever-increasing burden on urban transportation infrastructure. As cities around the world grapple with the challenges posed by burgeoning traffic volumes, there is a pressing demand for innovative and adaptive solutions.

In this context, my motivation converges with a broader societal need for intelligent traffic light systems. These systems have the potential to revolutionize urban transportation by dynamically managing traffic flows, reducing congestion, and mitigating environmental concerns. By harnessing the power of reinforcement learning and artificial intelligence, I aim to contribute to the development of intelligent traffic control systems that can serve as a model for cities worldwide.

Through this research endeavor, I aspire to make a meaningful impact by fostering more efficient and sustainable urban transportation systems. By optimizing traffic light control, I seek not only to alleviate the traffic woes in my homeland but also to offer a scalable solution that addresses the global imperative for intelligent traffic management.


% Objectives Section
\section{Objectives}
The key goals of this thesis are doing an in-depth examination of enhancing urban traffic flow using a strategy that incorporates a variety of different perspectives. These goals are meant to address the difficulties of traffic management, increase the level of realism in simulations, and evaluate the effectiveness of both cutting-edge algorithms and baseline controllers. The following is a list of the primary goals:

\subsubsection{Creation of Vake Map in SUMO}
The first objective is to develop a complex and representative urban traffic simulation environment within the Simulation of Urban MObility (SUMO) framework. This entails the creation of a detailed Vake map, capturing the intricacies of traffic infrastructure, including road networks, intersections, and traffic lanes. The map should accurately reflect the real-world urban environment under investigation.

\subsubsection{Generation of Realistic Traffic Patterns}
It is absolutely necessary to use actual traffic patterns in order to make the simulations as realistic as possible. In order to accomplish this goal, you will need to gather data on the traffic situation at the specified site in real time and painstakingly document the timing and behavior of the traffic lights. After that, the data that was acquired will be included into the simulation environment in order to imitate the actual circumstances of the traffic.

\subsubsection{Utilization of State-of-the-Art Algorithms}
The core of this research lies in the exploration, implementation and adaptation of state-of-the-art traffic signal control algorithms. The following algorithms will be employed:

\begin{itemize}
  \item \textbf{IDQN}: Implementing this deep reinforcement learning algorithm for traffic signal control, which has shown promise in optimizing signal timings.
  
  \item \textbf{IPPO}: Utilizing IPPO as another reinforcement learning algorithm to investigate its effectiveness in traffic management.
  
  \item \textbf{MPLIGHT}: Exploring MPLIGHT, a multi-phase control algorithm designed to adapt traffic signals dynamically.
  
  \item \textbf{FMA2C}: Investigating the potential of FMA2C for cooperative multi-agent traffic signal control.
\end{itemize}

\subsubsection{Comparison with Baseline Controllers}
To evaluate the performance of the selected state-of-the-art algorithms, this objective involves implementing and assessing the following baseline controllers:

\begin{itemize}
  \item \textbf{Fixed Time Control}: A traditional control strategy with fixed signal timings that do not adapt to real-time traffic conditions.
  
  \item \textbf{Max-Pressure Control}: Implementing this controller, which focuses on minimizing congestion by prioritizing the most congested lanes at intersections.
  
  \item \textbf{Greedy Control}: Assessing the performance of a basic greedy controller that makes decisions based on immediate traffic conditions.
\end{itemize}

\subsubsection{Comparative Analysis and Conclusions}
After the simulations and experiments have been run to completion, the data obtained from the various traffic signal control algorithms and baseline controllers will be meticulously examined and contrasted with one another. The goal is to arrive at significant findings on the effectiveness of each strategy in maximizing the flow of urban traffic. The purpose of this study is to shed light on the possibilities for intelligent traffic management systems to decrease environmental consequences, enhance efficiency, and alleviate congestion.

Through the accomplishment of these goals, the purpose of this thesis is to make a significant intellectual contribution to the subject of urban traffic optimization and to give actionable recommendations for improving traffic signal management systems in actual urban environments.
